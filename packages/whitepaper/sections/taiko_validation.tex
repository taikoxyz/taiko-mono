
\subsection{Block Validation and Mapping } \label{sec:filtermap}
The protocol filters proposed blocks using a \emph{txList Intrinsic Validity Function} $V^l$ on each block's txList $L$. If $V^l(L)$ returns \texttt{False}, the proposed block is dropped and ignored by L2 nodes; otherwise, the proposed block will map to an actual Taiko L2 block using the \emph{Block Mapping Function} $M(\dot{B})$.

\subsubsection{Validation} The txList Intrinsic Validity function requires:

\begin{enumerate}
\item The txList is RLP decodable into a list of transactions, and;
\item The number of transactions is no larger than the protocol constant $ K_{\mathrm{BlockMaxTxs}}$, and;
\item The sum of all transactions' gasLimit is no larger than the protocol constant $K_{\mathrm{BlockMaxGasLimit}}$, and;
\item Each and every transaction's signature is valid, i.e. it does not recover to the zero address.
\end{enumerate}

Formally, $V^l(L)$ is defined as:

\begin{eqnarray}
V^l(L) & \equiv & \texttt{NOERR}(T \equiv \texttt{RLP}'(L))  \quad \wedge  \\
\nonumber& & \lVert T \rVert \le K_{\mathrm{BlockMaxTxs}} \quad \wedge \\
\nonumber & & (\sum_{j = 0}^{\lVert T \rVert - 1}T[j]_g) \le K_{\mathrm{BlockMaxGasLimit}} \quad \wedge \\
\nonumber & & \prod_{j = 0}^{\lVert T \rVert - 1} (T[j]_g \ge K_{\mathrm{TxMinGasLimit}})\quad \wedge \\
\nonumber & & \prod_{j = 0}^{\lVert T \rVert - 1} (\texttt{NOERR}(\texttt{ECRECOVER}(T[j]) \ne 0) )
\end{eqnarray}

Where $\texttt{NOERR}(S)$ is a catch-error function that returns \texttt{False} if statement $S$ throws an error; $\texttt{RLP}'$ is the RLP decoding function;  $T_g$ is a transaction's gasLimit; 

The txList Intrinsic Validity function will be called on L2 and not on L1 because of the reasons explained in Section \ref{sec:txlist}.

\subsubsection{Mapping}

A proposed block where both $V^b(\dot{B})$ and $V^l(\dot{B}_L)$ hold true will map to an actual Taiko block.

Taiko blocks are identical to Ethereum blocks, as defined by the Ethereum Yellow Paper\cite{yellow-paper}:

\begin{eqnarray}
B_H & \equiv & (H_p, H_o, H_c, H_r, H_t, H_e, H_b, H_d, \\
\nonumber & & H_i, H_l,H_g, H_s, H_x, H_m, H_n) \\
B_U  & \equiv & [] \\
B & \equiv & (B_H, B_T, B_U)
\end{eqnarray}

Where $H_p$ is the block's parentHash, $H_o$ is the ommersHash, $H_c$ is the beneficiary, $H_r$ is the stateRoot, $H_t$ is the transactionsRoot, $H_e$ is the receiptsRoot, $H_b$ is the logsBloom, $H_d$ is the difficulty, $H_i$ is the block number, $H_l$ is the gasLimit, $H_g$ is the gasUsed, $H_s$ is the timestamp, $H_x$ is the extraData, $H_m$ is the mixHash, $H_n$ is the nonce; $B_T$ a series of the transactions; and $B_U$ is a list of ommer block headers but this list will always be empty for Taiko because there is no Proof-of-Work.

Transactions are identical to Ethereum transactions as defined by the Ethereum Yellow Paper\cite{yellow-paper}. However, only type 0 (legacy) transactions will be supported initially while EIP-1559 is disabled (but will be enabled in future versions).

A proposed block can only be mapped to a Taiko block in a \emph{Mapping Metadata} which is the world state $\boldsymbol{\sigma}$:

$$\boldsymbol{\sigma} \equiv (\boldsymbol{\delta}, h[1..256], d, i, \theta)$$

Where $\boldsymbol{\delta}$ is the state trie, $h[1..256]$ are the most recent 256 ancestor block hashes, $d$ is Taiko's chain ID, $\theta$ is the anchor transaction, and $i$ is the block number.

Now we can define the block mapping function $M$ as:

\begin{eqnarray}
M(B) & \equiv & M(H, T, U), \\
\nonumber & \equiv &  M(\boldsymbol{\delta}, h[1..256], {d}, i, \theta, \dot{B}, )  \\
\nonumber & \equiv & M(\boldsymbol{\delta}, h[1..256], {d}, i, \theta, C, L)
\end{eqnarray}

such that:

\begin{eqnarray}
& & \texttt{CHAINID}  = \quad \wedge \\
\nonumber& & \texttt{NUMBER} = {i} \quad \wedge \\
\nonumber& & U = [] \quad \wedge \\
\nonumber& & T =  \theta::V^t(\texttt{RLP}'(L)) \quad \wedge  \\
\nonumber& & H_p =  h(1) \quad \wedge \\
\nonumber& & H_o =   K_{\mathrm{EmptyOmmersHash}} \quad \wedge \\
\nonumber& & H_c =   C_c \quad \wedge \\
\nonumber& & H_d =   0 \quad \wedge \\
\nonumber& & H_i =   i \quad \wedge \\
\nonumber& & H_l =   C_l + K_{\mathrm{AnchorTxGasLimit}} \quad \wedge \\
\nonumber& & H_s =   C_s \quad \wedge \\
\nonumber& & H_x =   C_x \quad \wedge \\
\nonumber& & H_m =   C_m \quad \wedge \\
\nonumber& & (H_r, H_t, H_e, H_l, H_g) =   \Pi(\boldsymbol{\sigma}, (T_0, T_1, ...))
\end{eqnarray}

Where $\Pi$ is the block transition function; $::$ is the list concatenation operator; $V^t$ is the \emph{"Initial Tests of Intrinsic Validity"} function defined in the Transaction Execution section of the Ethereum Yellow Paper. To avoid confusion, in this document, we call $V^t$ the \emph{Metadata Validity} function.

$V^t(\texttt{RLP}'(L))$ yields a list of transactions that pass the tests; transactions that don't pass the tests are ignored and will not be part of the actual L2 block. Note that it is perfectly valid for $V^t(\texttt{RLP}'(L))$ to return an empty list.
