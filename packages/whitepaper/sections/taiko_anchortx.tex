\subsection{Anchor Transaction} \label{sec:anchoring}

The anchor transaction is a way for the protocol to make use of the programmability of the EVM (which we already need to be able to proof) to enforce certain protocol behavior. We can add additional tasks to anchor transactions to enrich Taiko's functionalities by writing standard smart contract code (instead of requiring more complicated changes to Taiko's ZK-EVM and node subsystems).

The anchor transaction is required to be the first transaction in a Taiko block (which is important to make the block deterministic). The anchor transaction is currently used as follows:

\begin{enumerate}
\item Persisting l1Height $C_a$ and l1Hash $C_h$, data inherited from L1, to the storage trie. These values can be used by bridges to validate cross-chain messages (see Section \ref{sec:bridges}).
\item Comparing $\rho_{i-1}$, the \textit{public input hash} stored by the previous block, with $\texttt{KEC}(i-1, d, h[2..256])$. The anchor transaction will throw an exception if such comparison fails. The protocol requires the anchor transaction to execute successfully and will not accept a proof for a block that fails to do so. Note that the genesis block has $\rho_0 \equiv \texttt{KEC}(0, d, [0,...,0])$.
\item Persisting a new public input hash

$$\rho_i \equiv \texttt{KEC}(i, d, h[1..255])$$ 

to the storage trie for the next block to use. This allows transactions, in the current and all following blocks, to access these public input data with confidence as their values are now covered by ZK-EVM's storage proof.
\end{enumerate}

With anchoring, the block mapping function $M$ can be simplified to:
\begin{eqnarray}
B & \equiv & (H, T, U), \\
\nonumber & \equiv &  M(\boldsymbol{\delta}, \theta, \dot{B}, )  \\
\nonumber & \equiv & M(\boldsymbol{\delta},  \theta, C, L)
\end{eqnarray}

\subsubsection{Construction of Anchor Transactions} All anchor transactions are signed by a \textit{Golden Touch} address with a revealed private key. 

Anchor transactions are constructed by Taiko L2 nodes as follows:

\begin{eqnarray}
& & \theta_x = 0 \quad \wedge \\
\nonumber& & \theta_n = \boldsymbol{\delta}[K_{\mathrm{GoldenTouchAddress}}]_n + 1 \quad \wedge \\
\nonumber& & \theta_p = 0 \quad \wedge \\
\nonumber& & \theta_g = K_{\mathrm{AnchorTxGasLimit}} \quad \wedge \\
\nonumber& & \theta_t = K_{\mathrm{GoldenTouchAddress}} \quad \wedge  \\
\nonumber& & \theta_v = 0 \quad \wedge  \\
\nonumber& & \theta_d = K_{\mathrm{AnchorTxSelector}}::C_a::C_h \quad \wedge  \\
\nonumber& & (\theta_r,\boldsymbol{\delta}_s) = \texttt{K1ECDSA}(\boldsymbol{\delta}, K_{\mathrm{GoldenTouchPrivateKey}})
\end{eqnarray}

Where \texttt{K1ECDSA} is the ECDSA\cite{ecdsa} signing function with the internal variable $k$ set to $1$, which guarantees the transaction's signature to only depend on the transaction data itself and is therefore deterministic.

According to the ECDSA's spec, when $k$ is $1$, $\theta_r$ must equal $\mathrm{G_x}$, the value of the x-coordinate of the base point on the SECP-256k1 curve. The \underline{TaikoL1} contract verifies this assertion.
